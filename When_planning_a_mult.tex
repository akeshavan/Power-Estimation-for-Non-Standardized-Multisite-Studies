When planning a multisite study, there is an emphasis on acquiring data from more sites because the estimated effect sizes from each site are sampled from a distribution and averaged. Understanding how much of the variance in the distribution is due to scanner noise as opposed to population heterogeneity is an important part of powering a study. For the purposes of this study, we estimated the effect size variability of Freesurfer-derived regional volumes, but this framework could be generalized to any T1-weighted segmentation algorithm, and any modality for which systematic errors are scaled. Scaling factor calibration of metrics resulted in higher absolute agreement of metrics between sites, which showed that the scaling factor variabilities for the ROIs in this study were accurate. The equation for power we outlined in this study along with our measurements of variability between sites should help researchers undestand the trade-off between protocol harmonization and sample size optimization, along with the choice of outcome metrics. Our statistical model and bias measurements enables collaboration between research institutions and hospitals when hardware and software adaptation are not feasible. We provide a comprehensive framework for assessing and making informed quantitative decisions for MRI facility inclusion, pipeline and metric optimization, and study power.