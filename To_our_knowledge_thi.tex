To our knowledge, this is the first study measuring scaling factors between sites with non-standardized protocols using a single set of subjects, and deriving an equation for power that takes this scaling into account via mixed modeling. This study builds on the work of \cite{fennema2007feasibility}, which investigated the feasibility of pooling retrospective data from three different sites with non-standardized sequences using standard pooling, mixed effects modeling, and fixed effects modeling. \cite{fennema2007feasibility} found that mixed effects and fixed effects modeling outperformed standard pooling. Our statistical model specifies how MRI bias between sites affects the cross-sectional mixed effects model, so it is limited to powering cross-sectional study designs. Jones and colleagues have derived sample size calculations for longitudinal studies acquired under heterogeneous conditions without the use of calibration subjects \cite{jones2013quantification}. This can be useful for studies measuring longitudinal atrophy over long time periods, during which scanners and protocols may change. For the cross-sectional case, the use of random effects modeling enables us to generalize our results to any protocol with acquisition parameters similar to those described here (primarily MPRAGE). If protocols change drastically compared to our sample of 3D MPRAGE-type protocols, a small set of healthy controls should be scanned before and after any major software, hardware, or protocol change so that the resulting scaling factors can be compared to the distribution of scaling factors ($CV_a$) reported in this study. A large $CV_a$ can severely impact the power of a multisite study, so it is important not to generalize the results in this study to non-MPRAGE sequences without validation. Potentially, new 3D-printed brain-shaped phantoms with similar regional contrast to noise ratios as human brains may become an excellent option for estimating $CV_a$.