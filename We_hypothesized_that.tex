We hypothesized that all differences in contrast and geometric distortion result in regional volumes that are consistently and linearly scaled from their true value. For a given region of interest (ROI), two mechanisms impact the final boundary definition: gradient nonlinearities cause distortion and, simultaneously, scanner and acquisition parameters modulate tissue contrast, adjusting the boundary on the order of $1-2\%$. By imaging 12 subjects in 20 different scanners using varying acquisition schemes, we were able to estimate the scaling factor for each regional volume at each site. We also defined a framework for calculating the power of a multisite study as a function of the scaling factor variability between sites. This enables us to both power a cross-sectional study, and to outline the conditions under which calibration is necessary. This framework can also indicate which regional volumes are sufficiently reliable to investigate using a multisite approach.