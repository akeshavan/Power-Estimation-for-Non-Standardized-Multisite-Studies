Towards this goal, there is a large body of literature addressing the correction of geometric distortions that result from gradient non-linearities. These corrections fall into two main categories: phantom-based deformation field estimation and direct magnetic field gradient measurement-based deformation estimation, the latter of which requires extra hardware and spherical harmonic information from the manufacturer \cite{fonov2010improved}. Calibration phantoms, such as the Alzheimer's Disease Neuroimaging Initiative (ADNI) \cite{gunter2009measurement} and LEGO® phantoms \cite{caramanos2010gradient}, have been used by large multisite studies to correct for these geometric distortions, which also affect regional volume measurements. These studies have outlined various correction methods that significantly improve deformation field similarity between scanners. However, the relationship between the severity of gradient distortion and its effect on regional volumes, in particular, remains unclear. In the case of heterogeneous acquisitions, correction becomes especially difficult due to additional noise sources. Gradient hardware differences across sites are compounded with contrast variation due to sequence parameter changes. In order to properly evaluate the reproducibility of brain segmentation algorithms, these phantoms should resemble the humain brain in size, shape, and tissue distribution. Droby and colleagues evaluated the stability of a post-mortem brain phantom and found similar reproducibility of volumetric measurements to that of a healthy control \cite{droby2015human}. In this study, we propose to measure between-site bias through direct calibration of regional volumes by imaging 12 common healthy controls at each site. Quantifying regional bias allows us to overcome between-site variability by optimizing sample size, rather than employing a phantom-based voxel-wise calibration scheme that corrects both contrast differences and geometric distortions. 