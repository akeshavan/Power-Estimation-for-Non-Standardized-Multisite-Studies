\subsection{Processing}
 A neuroradiologist reviewed all images to screen for major artifacts and pathology. The standard Freesurfer \cite{freesurferPaper} version 5.3.0 cross-sectional pipeline (recon-all) was run on each site's native T1-weighted protocol, using the RedHat 7 operating system on IEEE 754 compliant hardware. Both 1.5T and 3T scans were run with the same parameters (without using the -3T flag), meaning that the non-uniformity correction parameters were kept at the default values. All Freesurfer results were quality controlled by evaluating the cortical gray matter segmentation and checking the  linear transform to MNI305 space which is used to compute the estimated total intracranial volume \cite{buckner2004unified}. Scans were excluded from the study if the cortical gray matter segmentation misclassified parts of the cortex, or if the registration to MNI305 space was grossly innaccurate. Three scans were excluded for misregistration. Two exclusions were because of data transfer errors. Because of time constraints, some subjects were not able to be scanned. One of the 12 subjects could not travel to all the sites, and that subject was replaced by another of the same age and gender. The details of this are provided in the supplemental materials and the total number of scans is shown in tables \ref{tab:acquisition1} - \ref{tab:acquisition4}. 46 Freesurfer ROIs, including the left and right subcortical ROIs, from the aparc.stats tables, were studied. In this study we report on the ROIs relevant to the disease progression of MS, which include the gray matter volume (GMV), subcortical gray matter volume (scGMV), cortex volume (cVol), cortical white matter volume (cWMV), and the volumes of the lateral ventricle (LV), amygdala (amyg), thalamus (thal), hippocampus (hipp), caudate (caud). The remaining ROIs are reported in the supplemental materials. 