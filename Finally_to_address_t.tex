Finally, to address the concern about whether these scaling factors could apply to a disease population, we calculated scaling factors from 12 healthy controls and 14 MS patients between 2 different sequences at the UCSF scanner. The patients had a mean age of 51 years with standard deviation of 11 years, mean disease duration of 15 years with a standard deviation of 12 years, and mean Kurtzke Expanded Disability Status Scale (EDSS) \cite{Kurtzke_1983} score of 2.8 with a standard deviation of 2.2. The two sequences were an MPRAGE (see table \ref{tab:acquisition2} for parameters of site 18) and a 3D-FLASH sequence (TR=20ms, TE=4.92ms, flip angle=25 degrees, resolution=1mm isotropic). Because MS patients have lesions that could affect Freesurfer's tissue classification, all images were manually corrected for lesions on the T1-weighted images by a neurlogist after editing by Freesurfer's quality assurance procedure, which included extensive topological white matter corrections, white matter mask edits, and pial edits on images that were not lesion filled. Images that were still misclassified after thorough edits were removed from the analysis, because calibration factors need to be calculated from accurate segmentations.