Finally, to address the concern about whether these scaling factors could apply to a disease population, we calculated scaling factors from 12 healthy controls and 14 MS patients between 2 different sequences (3D-MPRAGE  versus 3D-FLASH) at the UCSF scanner (site 12). The patients had a mean age of 51 years with standard deviation of 11 years, mean disease duration of 15 years with a standard deviation of 12 years, and mean Kurtzke Expanded Disability Status Scale (EDSS) \cite{Kurtzke_1983} score of 2.8 with a standard deviation of 2.2. 

The accuracy of our scaling factor estimates depends on the accuracy of tissue segmentation, but the lesions in MS specifically impact white matter and pial surface segmentations. Because of the effect of lesions on Freesurfer's tissue classification, all images were manually corrected for lesions on the T1-weighted images by a neurologist after editing by Freesurfer's quality assurance procedure, which included extensive topological white matter corrections, white matter mask edits, and pial edits on images that were not lesion filled.  These manual edits altered the white matter surface so that white matter lesions were not misclassified as gray matter or non-brain tissue. The errors in white matter segmentations most typically occurred at the border of white matter and gray matter and around the ventricles. The errors in pial surface segmentations most typically occurred near the eyes (orbitofrontal) and the superior frontal or medial frontal lobes. Images that were still misclassified after thorough edits were removed from the analysis, because segmentations were not accurate enough to produce realistic scaling factor estimates.