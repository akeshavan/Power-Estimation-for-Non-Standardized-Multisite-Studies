Linear mixed models are common in modeling data from multisite studies because metrics derived from scanner, protocol, and population heterogeneity may not have uncorrelated error terms when modeled in a general linear model (GLM), which violates a key assumption \cite{garson2013fundamentals}. In fact, Fennema-Notestine and colleagues found that a mixed model, with scanner as a random effect, outperformed pooling data (via GLM)\cite{fennema2007feasibility} on a study on hippocampal volumes and aging. Since we are only interested in the effect of scanner-related heterogeneity, we assume that the relationship between the volumetrics and clinical factors of interest are the same at each site. This causes error terms to cluster by scanner and sequence type due to variation in field strengths, acquisition parameters, scanner makes, head coil configurations, and field inhomogeneities, to name a few \cite{cannon2014}. Linear mixed models, which include random effects and hierarchical effects, appropriately integrate observation-level data based on their clustering characteristics \cite{garson2013fundamentals}. The model we propose in this study is similar to a mixed model, with a multiplicative effect instead of an additive effect. Our goal is to incorporate an MRI bias-related term in our model in order to optimize sample sizes.