In this study we proposed a statistical model based on on the physics of MRI volumetric biases using the key assumption that biases between sites are scaled linearly. Variation in scaling factors could explain why a study may estimate different effect sizes based on the pulse sequence used. For example, \cite{streitburger2014impact} found significant effects of RF head coils, pulse sequences, and resolution on VBM results. The estimation of scaling factors in our model depends on good scan-rescan reliability. In our study, scan-rescan reliabilities for each scanner were generally $>0.8$ for Freesurfer-derived regional volumes. Volumes of cortex, cortical gray, subcortical gray, and cortical white matter parcellation had greater than 90\% reliability for all 20 sites. The subcortical regions and estimated total intracranial volume had an average reliability of over 89\%, however, some sites had much lower scan-rescan reliability. For example, the thalamus at sites 3 and 16 had test-retest reliabilities between 41 and 63 \%. This could be explained by the visual quality control process of the segmented images, which focused on the cortical gray matter segmentation and the initial standard space registration only, due to time restrictions. Visually evaluating all regional segmentations may be unrealistic for a large multisite study. On the other hand, Jovicich and colleagues \cite{jovicich2013brain} reported a low within-site ICC of the thalamus across sessions ($0.765 \pm 0.183$) using the same freesurfer cross-sectional pipeline as this study. The poor between-site reliability (61\%) of the thalamus is consistent with findings from \cite{schnack2010mapping}, in which a multisite VBM analysis showed poor consistency in that region. Other segmentation algorithms may be more robust for subcortical regions in particular. Using FSL's FIRST segmentation algorithm, Cannon and colleagues \cite{cannon2014} report a between-site ICC of the thalamus of 0.95, compared to our calibrated between-site ICC of 0.78. FSL's FIRST algorithm \cite{firstcitation} uses a Bayesian model of shape and intensity features to produce a more precise segmentation. Nugent and colleagues reported the reliability of the FIRST algorithm across 3 platforms. Their study of subcortical ROIs found a good scan-rescan reliability of 83\%, but lower between-site ICCs ranging from 57\% to 93\% \cite{firstreliability}. The LEAP algorithm proposed by Wolz and colleagues \cite{Wolz_2010} was shown to be extremely reliable with strong ICCs $>0.97$ for hippocampal segmentations \cite{Wolz_2014}. Another factor not accounted for in our segmentation results was the effects of partial voluming, which adds uncertainty to tissue volume estimates. In \cite{Roche_2014}, researchers developed a method to more accurately estimate partial volume effects using only T1-weighed images from the ADNI dataset. This approach resulted in higher classification accuracy between Alzheimer's disease (AD) patients and mild cognitively impaired (MCI) patients from normal controls (NL). Designing optimized pipelines that are robust for each site, scanner make, and metric, is outside the scope of this paper. However, Kim and colleagues have developed a robust technique for tissue classification of heterogeneously acquired data that incorporates iterative bias field correction, registration, and classification \cite{optimize}. Wang and colleagues developed a method to reduce systematic errors of segmentation algorithms relative to manual segmentations by training a wrapper method that learns spatial patterns of systematic errors \cite{Wang2011}. Methods such as those employed by Wang and colleagues may be preferred over standard segmentation pipelines when data acquisition is not standardized. Due to its wide range of acquisition parameters and size of the dataset, our approach could be used to evaluate such generalized pipelines in the future.