Test-retest reliabilities for each scanner were very good ($>0.8$) for Freesurfer derived regional volumes. Cortex volume, cortical gray, subcortical gray, and cortical white matter volumes had greater than 90\% reliability for all 20 sites. The subcortical regions and estimated total intracranial volume had an average reliability of over 89\%, however, some sites had much lower scan-rescan reliability. For example, the thalamus at sites 19 and 20 had test-retest reliabilities of 50 and 62 \%, respectively. This could be explained by the visual quality control process of the segmented images, which focused on the cortical gray matter segmentation only due to time restrictions. Visually evaluating all regional segmentations may be unrealistic for a large multisite study. Other segmentation algorithms may be more robust for subcortical regions in particular. The FIRST algorithm \cite{firstcitation} uses a Bayesian model of shape and intensity features for a more accurate segmentation. Nugent and colleagues reported the reliability of the FIRST algorithm across 3 platforms, and found a good scan-rescan reliability of 83\%, but the between-site ICCs ranged from 57\% to 93\%, based on the subcortical region of interest \cite{firstreliability}. Designing optimized pipelines that are robust for each site, scanner make, and metric, is outside the scope of this paper. However, Kim and colleagues have developed a robust technique for tissue classification of heterogeneously acquired data that incorporates iterative bias field correction, registration, and classification \cite{optimize}. Wang and colleagues developed a method to reduce systematic errors of segmentation algorithms relative to manual segmentations by training a wrapper method that learns spatial patterns of systematic errors \cite{Wang2011}. These methods may be preferred over standard segmentation pipelines when data acquisition is not standardized. The large range of acquisition parameters and size of this dataset will be useful to evaluate such generalized pipelines in the future.