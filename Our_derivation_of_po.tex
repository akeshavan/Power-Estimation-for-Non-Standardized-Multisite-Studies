Our derivation of power for a multisite study which was based on scaled, systematic error from MRI defines hard thresholds for the amount of acceptable scaling factor variability, $CV_{\alpha}$. Many conditions contribute to the $CV_{\alpha}$ cut-off, such as the total number of subjects, sites, effect size, and false positive rate. In Figure \ref{fig:cv_j}, we show the distribution of experimental $CV_{\alpha}$ values across all Freesurfer cortical and subcortical ROIs to compare with power curves of various sample sizes. Most of the $CV_{\alpha}$ values are below 10\%, and with enough subjects and sites, this falls well below the maximum acceptable $CV_{\alpha}$ value. However, we have shown one example of a hypothetical study with 2175 subjects and 23 sites, where directly calibrating volumes would be necessary, because the maximum acceptable $CV_{\alpha}$ is 3\%, and 90\% of the Freesurfer ROIs have a $CV_{\alpha}$ above this value. We have also validated our scaling factors by demonstrating that a leave-one-out calibration resulted in increased absolute agreement between sites compared to the original, uncalibrated values. To support this, we simulated multisite data using scaling factor estimates and their residual error, and found that the power curve matches the theoretical curve when power is at least 80\%.